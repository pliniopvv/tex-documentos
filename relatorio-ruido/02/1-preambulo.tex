\usepackage[utf8]{inputenc}
% set font arial
\usepackage[]{helvet}
\renewcommand{\rmdefault}{phv}
% set table colors
\usepackage[table]{xcolor}
%mesclando células vertical
\usepackage{multirow}
%colocando figuras
\usepackage{graphicx}
\usepackage{pdfpages}
\usepackage{float}
\usepackage{subfig}
% para plotagem de gráficos
\usepackage{pgfplots}   % pacote para uso do pgfplots
\pgfplotsset{width=1.0\linewidth,compat=1.11}
%\pgfplotsset{width=8cm,compat=1.11}
%\pgfplotsset{height=0.8\textheight,compat=1.11}
% pasta para imagens
\graphicspath{{temp/imagens/}}

%\usepackage{lipsum} %módulo para testes


%Configuração do ambiente Grafico
\newcommand{\dados}[2]{
	\addplot[color=#1] coordinates {
		#2
	};
}
\newenvironment{Grafico}{ %colocar os dados no meio e as legendas.
	\begin{tikzpicture}
	\begin{axis}[
	%ymin=0,
	minor tick num=4,
	enlarge x limits=false,
	every axis plot post/.append style=
	{mark=none},
	const plot,
	legend style={
		area legend,
		at={(0.5,-0.25)},
		anchor=north,
		legend columns=-1},
	ylabel=dB(A),
	xlabel=Segundos
	]
}{
\end{axis}
\end{tikzpicture}
}

%Nova configuração da página dos pontos
\newenvironment{Ponto}[1]{
	\addtocounter{#1}{1}
	\section{Resultados do ponto \arabic{#1}}
	\begin{SingleSpace}
		\begin{table}[h]
			\centering
			\caption{Resultado do ponto \arabic{#1}.}
		}{
	\end{table}
\end{SingleSpace}
}
\newenvironment{PontoTabela}{
	\begin{tabular}{|>{\centering\arraybackslash}m{1.7cm}|>{\centering\arraybackslash}m{2.0cm}|@{}>{\centering\arraybackslash}m{1.1cm}@{}|@{}>{\centering\arraybackslash}m{1cm}@{}|@{}>{\centering\arraybackslash}m{1.4cm}@{}|@{}>{\centering\arraybackslash}m{1.2cm}@{}|@{}>{\centering\arraybackslash}m{1.2cm}@{}|@{}>{\centering\arraybackslash}m{1.5cm}@{}|>{\centering\arraybackslash}m{4cm}|}
		\cline{1-3}
	}{
\end{tabular}
}
\newcommand{\fileName}[1]{
	\textbf{\cellcolor{black!25} Arquivo} & \multicolumn{4}{|c|}{\textbf{\cellcolor{black!25} #1}} & \multicolumn{4}{|c}{} \\ \cline{1-5}
}
\newcommand{\fileInit}[1]{
	Início & \multicolumn{4}{|c|}{#1} & \multicolumn{4}{|c}{}  \\ \cline{1-5}
}
\newcommand{\fileFim}[1]{
	Fim & \multicolumn{4}{|c|}{#1} & \multicolumn{4}{|c}{} \\ \hline
}
\newcommand{\fileInfo}[5]{
	Canal & Tipo & Peso & Un & LAeq Bruto & Lmin & Lmax & LAeq Tratado & Arredondamento conforme NBR 10151/2000 \\ \hline
	5016 & LAeq  & A & dB & #1 & #2 & #3 & #4 & \cellcolor{black!25} #5 \\ \hline
}

%Configurações da página dos pontos

%\newcommand{\grafico}[2]{\ref{#1\arabic{#1}#2}}
%\newcommand{\primeirografico}[1]{\ref{primeiro_grafico_#1}}
%\newcommand{\ultimografico}[1]{\ref{ultimo_grafico_#1}}
\newcommand{\ponto}[1]{\arabic{#1}}

%Configurações da página dos anexos
\newcounter{anexo}
\newcommand{\anexo}[1]{
	\addtocounter{anexo}{1}
	\newpage
	
	\
	
	\
	
	\
	
	\
	
	\
	
	\
	
	\
	
	\
	
	\
	
	\
	
	\
	
	\
	
	\
	
	\
	
	\
	
	\
	
	\begin{center}
		\textbf{ANEXO \arabic{anexo} - #1}
		\addcontentsline{toc}{chapter}{ANEXO \arabic{anexo} - #1}
	\end{center}
	\newpage
}

%\newcommand{\primeiratabela}[1]{\ref{primeira_tabela_#1}}
%\newcommand{\ultimatabela}[1]{\ref{ultima_tabela_#1}}
%\newcommand{\setprimeiratabela}[1]{\label{primeira_tabela_#1}}
%\newcommand{\setultimatabela}[1]{\label{ultima_tabela_#1}}
%\newcommand{\setprimeirografico}[1]{\label{grafico:#1:\ponto{#1}}}
%\newcommand{\setultimografico}[1]{\label{grafico:#1:\ponto{#1}}}

%Configurações da página das fotos
\newcommand{\primeirafoto}{\ref{foto_primeira}}
\newcommand{\ultimafoto}{\ref{foto_ultima}}

\newcommand{\DuasFotosDoPonto}[5]{% \DuasFotosDoPonto{NomeDaPrimeiraFoto.jpg}{Descrição da primeira foto}{NomeDaSegundaFoto.jpg}{Descrição da segunda foto}{Marcar qual foto? 0=Nenhuma|1=Primeira|2=Segunda}
	\begin{minipage}{\linewidth}
		\centering
		\begin{minipage}{0.45\linewidth}
			\begin{figure}[H]
				\includegraphics[width=8cm]{#1}
				\caption{#2}
				\ifnum #5 = 1
				\label{foto_primeira}
				\else
				\fi
			\end{figure}
		\end{minipage}
		\hspace{0.05\linewidth}
		\begin{minipage}{0.45\linewidth}
			\begin{figure}[H]
				\includegraphics[width=8cm]{#3}
				\caption{#4}
				\ifnum #5 = 2
				\label{foto_ultima}
				\else
				\fi
			\end{figure}
		\end{minipage}
	\end{minipage}
}

\newcommand{\UmaFotoDoPonto}[3]{% \UmaFotoDoPonto{NomeDaFoto.jpg}{Descrição da Foto}{Marcar a foto como ultima? 0=Não|1=Sim}
	\begin{figure}[H]
		\centering
		\includegraphics[width=8cm]{#1}
		\caption{#2}
		\ifnum #3 = 1
		\label{foto_ultima}
		\else
		\fi
	\end{figure}
}

% CONFIGURANDO CABEÇALHO
\makepagestyle{LRA}
\makeoddhead{LRA}{}{\scriptsize{\underline{Doc. XXXX-2016 - Laudo de Avaliação Ambiental de Ruído - CERAMICA TIJUCANA LTDA - EPP} \\}}{\scriptsize{\underline{Página \thepage\ de \thelastpage}}}
%\makeevenhead{LRA}{}{\scriptsize{\underline{RA 0086-2014 Laudo de Avaliação Ambiental de Ruído – Cerâmica União Ituiutaba Ltda.  Outubro / 2014} \\}}{\scriptsize{\underline{Página \thepage\ de \thelastpage}}}
%\addtocounter{page}{1} % ADICIONANDO +1 PÁGINA PARA CONTAR A CAPA DO TRABALHO
% CONFIGURANDO CABEÇALHO - FIM

% CONFIGURANDO O ESTILO DO CABEÇALHO
\makechapterstyle{section}{
	\renewcommand{\printchaptername}{}
	\renewcommand{\chapternamenum}{}
	\renewcommand{\chaptitlefont}{\normalsize}
	\renewcommand{\chapnumfont}{\normalfont\normalsize\bfseries}
	\renewcommand{\printchapternum}{\chapnumfont \thechapter\space}
	\renewcommand{\afterchapternum}{}
	\renewcommand{\afterchapskip}{\onelineskip}
	\renewcommand{\beforechapskip}{0pt}
}
\renewcommand{\ABNTEXsectionfontsize}{\normalsize}

\chapterstyle{section}
% CONFIGURANDO O ESTILO DO CABEÇALHO - FIM


















%\newenvironment{TabelaResumo}{
%\begin{table}[h]
%\centering
%\begin{scriptsize}
%}{
%\end{scriptsize}
%\end{table}
%}
%\newenvironment{DadosResumo}{
%\begin{tabular}{|>{\centering\arraybackslash}m{1.3cm}|>{\centering\arraybackslash}m{1.3cm}|>{\centering\arraybackslash}m{1.3cm}|>{\centering\arraybackslash}m{1.3cm}|>{\centering\arraybackslash}m{1.3cm}|>{\centering\arraybackslash}m{1.3cm}|>{\centering\arraybackslash}m{1.3cm}|>{\centering\arraybackslash}m{1.3cm}|>{\centering\arraybackslash}m{1.3cm}|}
%}{
%\hline
%\end{tabular}
%}
%\newcommand{\titleResumo}[3]{
%
%\hline
%\multicolumn{9}{|c|}{\cellcolor{black!25} \textbf{#1}} \\
%\multicolumn{9}{|c|}{\cellcolor{black!25} \textbf{#2}} \\
%\hline
%Número do Ponto & Número da Tabela & LAeq  Dados brutos dB(A) & LAeq Dados Tratados dB(A) & Lc (componentes tonais/impulsivos) dB(A) & Lra (Ruído Ambiente) dB(A) & NCA dB(A) da 10151/2000 & NCA Corrigido em relação ao Lra dB(A) & Comparação do LAeq Dados tratados com NCA (ou NCA corrigido) (NBR 10151/2000 ) \\ \hline
%\\
%}
%%Primeira linha e suas variações
%\newcommand{\firstLineResumoAprovedByNormalNCA}[6]{
%\nextRow & \nextDot & #1 & \cellcolor{black!25} #2 &\cellcolor{black!25}  #3 & #4 & \multirow{\tablerows}{*}{\textcolor{blue}{#5}} &\cellcolor{black!25} #6 & \textcolor{blue}{Atende} \\ \cline{1-6} \cline{7-9}
%}
%\newcommand{\firstLineResumoAprovedByLra}[6]{
%	\nextRow & \nextDot & #1 & \cellcolor{black!25} #2 &\cellcolor{black!25}  #3 & \textcolor{red}{#4} & \multirow{\tablerows}{*}{\textcolor{blue}{#5}} & \cellcolor{black!25} #6 & \textcolor{blue}{Atende} \\ \cline{1-6} \cline{7-9}
%}
%\newcommand{\firstLineResumoReprovedByNormalNCA}[6]{
%	\nextRow & \nextDot & #1 & \cellcolor{black!25} #2 &\cellcolor{black!25}  #3 & #4 & \multirow{\tablerows}{*}{\textcolor{blue}{#5}} & \cellcolor{black!25} #6 &  \textcolor{red}{Não Atende} \\ \cline{1-6} \cline{7-9}
%}
%% Próximas linhas e suas variações
%\newcommand{\nextLineAprovedByNormalNCA}[5]{
%	\nextRow & \nextDot & #1 &\cellcolor{black!25} #2 &\cellcolor{black!25} #3 & #4 & & \cellcolor{black!25} \textcolor{blue}{#5} & \textcolor{blue}{Atende} \\ \cline{1-6} \cline{7-9}
%}
%\newcommand{\nextLineAprovedByLra}[5]{
%	\nextRow & \nextDot & #1 &\cellcolor{black!25} #2 &\cellcolor{black!25} #3 & \textcolor{red}{#4} & &\cellcolor{black!25} \textcolor{blue}{#5} & \textcolor{blue}{Atende} \\ \cline{1-6} \cline{7-9}
%}
%\newcommand{\nextLineReprovedByNormalNCA}[5]{
%	\nextRow & \nextDot & #1 &\cellcolor{black!25} #2 &\cellcolor{black!25} #3 & #4 & &\cellcolor{black!25} \textcolor{blue}{#5} & \textcolor{red}{Não Atende} \\ \cline{1-6} \cline{7-9}
%}