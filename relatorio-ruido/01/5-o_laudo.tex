
\newpage
\begin{SingleSpace}
\textbf{LAUDO DE AVALIAÇÃO AMBIENTAL DE RUÍDO}
\addcontentsline{toc}{chapter}{LAUDO DE AVALIAÇÃO AMBIENTAL DE RUÍDO}

\begin{flushright}
Uberlândia, \today .
\end{flushright}

\textbf{OBJETIVO}
\addcontentsline{toc}{section}{OBJETIVO}

Avaliar os níveis de pressão sonora emitidos pelo empreendimento Cerâmica União Ituiutaba Ltda., bem como o impacto gerado nos ambientes vizinhos.

Esta avaliação de ruídos fornece informações sobre a necessidade de implementação de medidas mitigadoras para reduzir o nível de pressão sonora aos padrões aceitáveis e/ou determinar procedimentos operacionais que possibilitem o funcionamento do empreendimento. \\

\textbf{CARACTERIZAÇÃO DO EMPREENDIMENTO EM ANÁLISE} \\	
\textbf{Amostragens realizadas em:}10 de Outubro de 2016 \\
\textbf{Empresa:} CERAMICA TIJUCANA LTDA - EPP \\
\textbf{Logradouro:} R CARLOS MARQUEZ DE ANDRADE\textbf{Nº:} 600\textbf{Cidade:} ITUIUTABA - MG \textbf{CEP:} 38.301-186\\
\textbf{CNPJ:} 20.299.568/0001-62 \textbf{IE:} 0 \\
\textbf{Fone:}  \\

\textbf{METODOLOGIA UTILIZADA NO TRABALHO}
\addcontentsline{toc}{section}{METODOLOGIA UTILIZADA NO TRABALHO}

O trabalho ora proposto considerou as seguintes fases:

\begin{enumerate}
\item Análise da legislação e normatizações sobre ruído;
\item Observação das fontes de ruídos no local, reconhecimento da área e pré-monitoramento com o intuito de definir os pontos de amostragem;
\item Realização das amostragens de acordo com procedimento interno;
\item Confecção de um croqui para identificação dos pontos de amostragem.
\item Tratamento e armazenamento dos dados;
\item Confecção do relatório, análises e conclusões baseadas nas normatizações vigentes.
\end{enumerate}

\textbf{SOBRE AS AMOSTRAGENS}
\addcontentsline{toc}{chapter}{SOBRE AS AMOSTRAGENS} \\

As amostragens foram realizadas conforme preconizado pela NBR 10151 / 2000, no perímetro externo do empreendimento, com o equipamento afastado a pelo menos 2 metros das divisas e a pelo menos 1,2 metros acima do nível do solo.

Cada amostragem foi realizada durante um período de três minutos, registrando o nível de pressão sonora de segundo em segundo, gerando um resultado final único para o ponto analisado. Foi utilizado medidor de ruído classe 1, ligado em escala de compensação A e respostas de leitura rápida.

\textbf{Não foram necessárias correções uma vez que não foi constatada a presença de ruídos com componentes tonais ou de caráter impulsivo.}

%\textbf{Não foram realizadas amostragens noturnas uma vez que o empreendimento não opera neste período.}

\end{SingleSpace}



