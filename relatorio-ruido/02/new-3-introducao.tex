\newpage
\chapter{\textbf{OBJETIVO}}

O Presente documento apresenta os resultados das medições de ruído na área de entorno do empreendimento \nomeEmpresa., situado em \cidadeSituada, realizadas de acordo com a ABNT NBR 10.151/2019 – versão corrigida em 2020. \\

As amostragens foram realizadas em 27 de abril de 2020.

\section{\textbf{INTRODUÇÃO}} 

O termo poluição sonora é aplicado a todo ruído que pode causar danos à saúde humana ou animal. O ruído se caracteriza pela existência de muitas amplitudes e frequências ocorrendo ao mesmo tempo de maneira não harmônica, enquanto que o som se caracteriza por poucas amplitudes e frequências, geralmente harmônicas. Ambos só têm sentido quando são captados por um ouvido humano ou de um animal. \\
Se o som ou o ruído é caracterizado por deslocamentos das partículas de um meio elástico em relação a suas posições de equilíbrio, as compressões e expansões do meio causam flutuações de pressão. Como essas flutuações ocorrem devido à propagação de um som, recebem a denominação de pressão sonora.

\section{\textbf{DEFINIÇÕES}}

\textbf{Decibel ponderado A}: Intensidade de som medida na curva de ponderação “A” utilizada para avaliações das reações humanas ao ruído. \\
\textbf{NPS}: Nível de Pressão Sonora.  \\
\textbf{Nível de Pressão Sonora Equivalente (LAeq)}: Nível obtido a partir do valor médio quadrático da pressão sonora referente a todo o intervalo de medição (medido em decibel com ponderação em “A”). \\
\textbf{Nível Sonoro Corrigido (LR)}: Nível medido ou calculado, ao qual tenha sido adicionada uma correção. \\
\textbf{Ruído}: Termo associado a sons que podem causar incômodos, sendo indesejáveis ou não inteligíveis. \\
\textbf{Som}: Flutuações de pressão. \\
\textbf{Som Contínuo}: Som presente durante todo o período de observação e que não é som intermitente nem um som impulsivo. \\
\textbf{Som Específico}: Parcela do som total que pode ser identificada e que está associada a uma determinada fonte. Sinônimo das terminologias de “Ruído de Fonte Específica” e “Ruído de Fonte”. É o objeto de medição. \\
\textbf{Som Flutuante}: Som contínuo cujo nível de pressão sonora, durante o período de observação, varia significativamente.\\
\textbf{Som de Impacto}: Som resultante do impacto entre materiais. \\
\textbf{Som Impulsivo}: Som caracterizado por impulsos de pressão sonora de duração inferior a 1s. \\
\textbf{Som Intermitente}: Som que ocorre apenas em certos intervalos de tempo, regulares ou não, em que a duração de cada um é superior a 1s. \\
\textbf{Som Intrusivo}: interferência sonora alheia ao objeto de medição. \\
\textbf{Som Residual}: Som remanescente do som total em uma dada posição e em uma dada situação, quando são suprimidos os sons específicos em consideração. Sinônimo das terminologias “Ruído de Fundo” e “Ruído Ambiente”. \\
\textbf{Som Tonal}: Som caracterizado por uma única componente de frequência ou por componentes de banda estreita que se destacam em relação às demais componentes.  \\
\textbf{Som Total}: Som existente em uma dada situação e em um dado instante. Considera a contribuição de todas as fontes sonoras. Será aqui tratado como Ruído Total. \\

\section{\textbf{Ruído X Som}}
A vibração mecânica de um corpo produz deslocamentos oscilatórios das partículas do meio circundante. Quando estas oscilações se propagam até os ouvidos, provocam a oscilação dos tímpanos e, por um mecanismo interno de transmissão, estimulam os nervos auditivos que, por sua vez, transmitem ao cérebro uma sensação percebida como som. Quando esses diversos movimentos oscilatórios se combinam e produzem um movimento resultante, cuja oscilação não se dê de forma harmônica, tem-se o que é chamado de ruído. \\
A fronteira entre som e ruído não pode ser definida com precisão, pois, cada indivíduo apresenta uma reação diferente ao som ou ao ruído, que depende dentre outros fatores, do estado emocional do indivíduo e sua personalidade. O ruído é associado a uma sensação não prazerosa.


\section{\textbf{Curvas de Ponderação}}
Devido ao fato de que o ouvido humano não é igualmente sensível ao som em todo o espectro de frequência, surgiram as chamadas curvas de ponderação. Um ser humano exposto a dois ruídos iguais em intensidade, porém distintos em frequência, terá uma sensação auditiva diferente para cada um deles. Um som de baixa frequência é geralmente menos perceptível do que um de alta frequência.  \\
Várias curvas foram então propostas na tentativa de se fazer com que os níveis sonoros captados pelos medidores fossem devidamente corrigidos para assemelharem-se à percepção do som pelo ouvido humano. Essas curvas de ponderação foram designadas pelas letras A, B, C, D, etc.


\section{\textbf{Nível Sonoro Equivalente} }
O potencial de danos à audição de um dado ruído depende não somente de seu nível, mas também de sua duração. Normalmente, os níveis de ruído podem variar durante um determinado intervalo de tempo. O nível sonoro equivalente (Leq) é um nível constante que equivale, em termos de energia acústica, aos níveis variáveis do ruído, durante o período de medição. Assim, é definido um valor único, chamado nível equivalente de pressão sonora, que é o nível sonoro médio integrado durante um intervalo de tempo. O valor é expresso em dB. \\
A expressão mostra que o nível equivalente é representado então por um valor constante que, durante o mesmo tempo T, resultaria na mesma energia acústica produzidas pelos valores instantâneos variáveis de pressão sonora. 
Portanto, um nível equivalente Leq tem o mesmo potencial de lesão auditiva que um nível variável considerado no mesmo intervalo de tempo. Os critérios para lesão permitem essa equivalência até aproximadamente 115 dB de nível máximo, a partir do qual pode ocorrer lesão com exposição de curta duração.


