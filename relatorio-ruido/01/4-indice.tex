\newpage
\renewcommand{\contentsname}{{\large \textbf{ÍNDICE}}}
\tableofcontents*

\

\textbf{LISTA DE SIGLAS E SIGNIFICADOS DA TABELA} \\
\addcontentsline{toc}{chapter}{LISTA DE SIGLAS E SIGNIFICADOS DA TABELA}

{\noindent
	\textbf{LAeq} - Nível de pressão sonora equivalente. \\
	\textbf{LAeq B} - LAeq Bruto - dados originais no momento da medição. \\
	\textbf{LAeq T} - LAeq Tratado - dados brutos excluindo-se as interferências externas pontuais ocorridas no momento da medição (carros, caminhões, aves, animais, entre outros, registrados pelo medidor, no instante da medição) com  efetuação do arredondamento simétrico do dado encontrado. \\
	\textbf{Lc} - Nível de ruído corrigido, aplicado para ruído com características especiais (tonais ou impulsivos), quando existir a presença desse ruído, Lc deve ser comparado com NCA. \\
	\textbf{Lra} - Nível de ruído ambiente - valor obtido após o tratamento e interpretação dos dados de campo. \\
	\textbf{NCAp} - Nível de Critério de Avaliação conforme estabelece a NBR 10151/2000
	\textbf{NCAc} - NCA corrigido em relação ao Lra - De acordo com a NBR 10151/2000, item 6.2.4, “Se o nível de ruído ambiente Lra, for superior ao valor da tabela 1 para a área e o horário em questão, o NCA assume o valor do Lra”. \\
	\textbf{Comparação}* - Comparação do LAeq Dados tratados com NCA (ou NCA corrigido) (NBR 10151/2000 ).
	\textbf{Atende} - quando o ruído emitido pelo empreendimento for menor do que o exigido pela legislação (NCA). \\
	\textbf{Não atende} - quando o ruído emitido pelo empreendimento for maior do que o exigido pela legislação (NCA) \\}